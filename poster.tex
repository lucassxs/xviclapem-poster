%%%%%%%%%%%%%%%%%%%%%%%%%%%%%%%%%%%%%%%%%%%%%%%%%%%%%%%%%%%
% Latex template for abstracts
%%%%%%%%%%%%%%%%%%%%%%%%%%%%%%%%%%%%%%%%%%%%%%%%%%%%%%%%%%%
% Formulario latex para resumos
%%%%%%%%%%%%%%%%%%%%%%%%%%%%%%%%%%%%%%%%%%%%%%%%%%%%%%%%%%%

\documentclass{article}
\usepackage[utf8]{inputenc}
\usepackage{amsmath,amsthm,amsfonts,amssymb,amscd}


\begin{document}
\thispagestyle{empty}
\newcommand{\titulo}[1]{{\large\bf #1}}
\newcommand{\nome}[1]{{\large #1}}
\newcommand{\instituicao}[1]{#1}
\newcommand{\resumo}[1]{#1}
\newcommand{\ack}[1]{\textbf{Acknowlegments:} #1}
\newcommand{\funding}[1]{\textbf{Funding:} #1}

\renewcommand{\refname}{\normalsize References}

%%%
% Write the title of the seminar-activity 
% Indique o titulo do seminario-atividade 
%%%
\noindent
\titulo{Title of the talk/poster contribution}

\bigskip

%%%
% Write your own name
% Indique o seu nome
%%%
\noindent
\nome{Name of the speaker}\\
\noindent

%%%
% Write your institutional affiliation
% Indique a sua instituicao de origem
%%%
\noindent
\instituicao{Name of the institution of the speaker}\\

%%%
% Write a short abstract of your talk (50-100 words)
% Escreva um resumo da sua palestra (50-100 palavras)
%%%

\noindent
\resumo{\newline 
In this talk...
}\\

\bigskip
%------
% Insert acknowledgments and information
% regarding funding at the end of the last
% section, i.e., right before the bibliography.
%------

\noindent
\ack{I thank... }\\

 
\noindent
\funding{This work was  supported by ... }


\begin{thebibliography}{99}%(if needed)
\normalsize
\bibitem{?} {\sc { Surname, Name and Surname, Name}}: {\it title of article}, { Journal's name, number, issue, pages range (year).}

\end{thebibliography}

\end{document}