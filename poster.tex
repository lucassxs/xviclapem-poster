%%%%%%%%%%%%%%%%%%%%%%%%%%%%%%%%%%%%%%%%%%%%%%%%%%%%%%%%%%%
% Latex template for abstracts
%%%%%%%%%%%%%%%%%%%%%%%%%%%%%%%%%%%%%%%%%%%%%%%%%%%%%%%%%%%
% Formulario latex para resumos
%%%%%%%%%%%%%%%%%%%%%%%%%%%%%%%%%%%%%%%%%%%%%%%%%%%%%%%%%%%

\documentclass{article}
\usepackage[utf8]{inputenc}
\usepackage{amsmath,amsthm,amsfonts,amssymb,amscd}


\begin{document}
\thispagestyle{empty}
\newcommand{\titulo}[1]{{\large\bf #1}}
\newcommand{\nome}[1]{{\large #1}}
\newcommand{\instituicao}[1]{#1}
\newcommand{\resumo}[1]{#1}
\newcommand{\ack}[1]{\textbf{Acknowlegments:} #1}
\newcommand{\funding}[1]{\textbf{Funding:} #1}

\renewcommand{\refname}{\normalsize References}

%%%
% Write the title of the seminar-activity 
% Indique o titulo do seminario-atividade 
%%%
\noindent
\titulo{WeatherMaringa - R Package}

\bigskip

%%%
% Write your own name
% Indique o seu nome
%%%
\noindent
\nome{Lucas Stefano Xavier de Sousa}\\
\noindent

%%%
% Write your institutional affiliation
% Indique a sua instituicao de origem
%%%
\noindent
\instituicao{State University of Maringá}\\

%%%
% Write a short abstract of your talk (50-100 words)
% Escreva um resumo da sua palestra (50-100 palavras)
%%%

\noindent
\resumo{\newline 
In this poster is presented the package developed in R "WeatherMaringa", 
whose purpose is to collect data from the INMET(National Institute of Meteorology) or 
ECPM(Main Climatological Station of Maringá) databases of the city of Maringá 
in the State of Paraná in Brazil through an API developed in R using 
the "Plumber" package, where the data are updated daily at 11:00 PM. 
In the operation of the package, the variables of the package 
perform through functions: descriptive analyses and other 
probabilistic statistical functions; and access to the INMET or ECPM database 
alternately.
}\\

\bigskip
%------
% Insert acknowledgments and information
% regarding funding at the end of the last
% section, i.e., right before the bibliography.
%------

\noindent
\ack{I thank to my advisor, professor Brian Alvarez Ribeiro de Melo, 
for his dedicated guidance throughout this challenging journey.}\\

 
\noindent
\funding{This work was  supported by ... }


\begin{thebibliography}{99}%(if needed)
\normalsize
\bibitem{?} {\sc { Surname, Name and Surname, Name}}: {\it title of article}, { Journal's name, number, issue, pages range (year).}

\end{thebibliography}

\end{document}